%\documentclass[11pt,twocolumn]{article}
\documentclass{article}
\begin{document}
\title{Is State of the Art Good Enough\\for Power Measurements}
\author{David DeBonis\\Sandia National Laboratories\\\\
Dorian Arnold\\University of New Mexico}
\renewcommand{\today}{July 22, 2013}
\maketitle
\section{Abstract}
\section{Introduction}
\section{Related Work}
Previous work which have targeted live data collection rather than
simulation have shown that thermal research on advanced aggresively-
clock-gated super-scalar processors is problematic~\cite{isci2003}.  In
that study the total power is gathered live through inductive clamp
with component power estimated based on performance counters which
are then correlated using Bayesian similarity matrices.  In our
work we gather our power values directly without the need for
probabolistic methods since we achieve component level granularity
directly.

Component level profiling was achieved in another study utilizing
ten multimeters per node, each monitoring power over a shunt resistor
inserted through ATX extension cables~\cite{feng2005}.  While their results
gathered accurate per node power data, the collection had to be made
through individual collection over each node through multiple passes.
The application under test also needed to be instrumented directly to
control the power gathering.  Our work allows us to collect concurrently
over all nodes regardless of scale in situ, avoiding the need to
manually retool a particular node in the system.

The implementation of a scalable power measurement framework was
presented utilizing board level interface exploitations to measure
power usage~\cite{laros2009}.  Though a great proof-of-concept and of
utility, the accuracy proved rather poor at +/- 2 amperes.  The
concept of Application Power Signatures was introduced and the
first quantitative study of OS noise was performed.  Many of the
concepts from that work were leveraged in our study.

A low-cost power monitoring device that can operate inside
commodity computing systems was introduced~\cite{bedard2011}.  In their
study it was noted that the use of AC monitors like PowerEgg and
WattsUp were inadequate to capture fast variations in the DC
load of the supply.  Through the use of sense resistors, individual
DC power rails were monitored with high resolution.  There is
a cost, of computational overhead on the host and temperature
rise of up to 20C on the device with a 10 amp load.  Our
presented work is passive to both the host and power rails
through the use of a seperate embedded processor with its own
communication infrastructure and inductive based monitoring.
\section{Methods}
\subsection{System Level Data Collection}
The power monitoring system provides centralized logging
of distributed clustered node power usage.

A daemon running on the top level node acts as a proxy for
receiving power information from individual agents running
on each of the PI modules at all times.  These agents can
individually be configured through peer-to-peer communication
between the PI and any node within the system (i.e. login node,
compute node, top level node, etc.).  The ability to configure
an agents sample-rate and individual sensor port state
(collecting or not) is exposed through a peer-to-peer
communication protocol over TCP/IP posix sockets.  Each agent
periodically (dependent on its configured sample rate) calls
down to the device to extract sensor port values which are
communicated to the proxy daemon.  Other topological configurations
and communication patterns are available but not presented within
this work.

The proxy daemon aggregates all of the agents log messages into
a flat file that forms the base data to be analyzed post-process.
Fine grained information down to the sample are retained without
data thinning to avoid any loss of detail.  Timing information
(which is regulated via the local cluster NTP) is attained down
to microseconds and power values to milliamps, millivolts, and
milliwatts are recorded.

A post-processing analysis suite is used to partition the
collection based on PI node and sensor port into individual
files for further visualization and mining.  Plots of the
partitioned files are also generated for fast visual analysis
of a given collection run.  A summary statistical analysis is
performed over all of the PI nodes and sensor ports to distille 
range and average values for amperage, voltage, and wattage along
with running time and power consumed.  A plot of this course-grained
information is generated for visual inspection.
\section{Experiments}
\subsection{Fidelity}
\subsection{Accuracy}
\section{Conclusions}
A key differentiator of PowerInsight is its out-of-band communication,
allowing for neither performance or power impact on system components.
We have shown that by using a seperate computing device to perform the
collection, calculations, and aggregation of power information that the
CPU can essentially offload all but what it is interested in (TBD -
could mention producer/subscriber model vs. polled).  In addition,
electically it is outside of the system since power is measured through
induced signaling.
\section{Future Work}
\section{Acknowledgements}
I would like to thank James Larros and Kevin Pedretti for their
support of this research.
\bibliographystyle{plain}
\bibliography{power}
\end{document}
